\documentclass[9pt,twocolumn]{article}
\usepackage[english]{babel}
\usepackage{listings}
\usepackage{hyperref}
\usepackage[edges]{forest}
\usepackage[backend=biber]{biblatex}
\bibliography{paper.bib}

% reproducible outputs
\pdfinfoomitdate=1
\pdftrailerid{}

\hyphenpenalty=10000
\lstset{
  basicstyle=\small\ttfamily,
  columns=flexible,
  breaklines=true,
  breakatwhitespace=true
  }
  \usepackage[letterpaper,top=2cm,bottom=2cm,left=3cm,right=3cm,marginparwidth=1.75cm]{geometry}
  \newcommand{\q}[1]{``#1''}

  \title{dp3: multimodal log database}
  \author{
    Wyatt Alt
    \texttt{wyatt.alt@gmail.com}
    \and
    Jainil Ajmera
    \texttt{jainil@berkeley.edu}
    }

    \begin{document}
    \maketitle

    \begin{abstract}
      dp3 is a multimodal log database for robotics data infrastructure, with goals
      to support low-latency playback, timeseries analytics, precomputed
      multigranular field-level statistical summaries, and efficient integration with
      distributed compute under a single system.

      The first section of this paper will introduce key concepts, background,
      and challenges in the multimodal logging domain. The second section will
      present the architecture of dp3.
    \end{abstract}

    \section{Introduction}
    dp3 is a row-oriented multimodal log database structured around a
    time-partitioned copy-on-write tree, which is stored in S3-compatible
    storage and fronted by an in-memory cache of inner tree nodes. It is
    designed for storage and retrieval of robotics data with three \q{user
    personas} in mind:

    The \textbf{robotics developer}, whose primary workload consists of
    alternating between searches for interesting log events and doing
    replay-based analysis of those events using a playback visualization tool,
    or possibly replaying logs through their software stack.

    The \textbf{ML developer}, whose primary workload consists of large-scale compute
    jobs against narrow selections of topics.

    The \textbf{data infrastructure admin}, who is responsible for administering the
    database and satisfying the other two customers in a cost-effective and
    maintainable manner.

    One person may of course assume multiple personas.

    For the robotics developer, dp3 provides low-latency playback for data recorded
    by a \q{producer} from a point in time on a selection of \q{topics}, as well as a
    query language for ergonomic expression of the \q{as-of join} conditions
    allowing the user to quickly pivot between searching for interesting events,
    and replaying those events in full context.

    For the ML developer, dp3 supports both client/server and embedded modes of
    operation. The query executor may be embedded directly into jobs so that jobs
    can operate directly against S3. This isolates heavy and bursty ML-related read
    workloads from the comparatively lighter-weight streaming playback and
    summarization workloads supported by the database server, and reduces the cost
    and complexity of infrastructure management by eliminating the need to scale a
    database deployment in step with the job workloads.

    dp3 minimizes operational burden for the data infrastructure by offloading
    primary storage and index structure to S3 and allowing the admin to use S3
    tooling (lifecycle policies, retention policies, etc.) to govern the garbage
    collection of the tree. To keep users from being exposed to deleted
    objects, the admin must ensure the tree storage is always truncated ahead
    of the physical deletion schedule. Since dp3 supports any S3-compatible
    storage, it can be deployed in on-premise environments in situations where
    cloud offload bandwidth is limited at the point of ingestion.

    dp3 is initially targeting a smooth integration experience with the ROS
    \cite{Quigley09} ecosystem and the MCAP \cite{MCAP} format, to become a viable
    \q{plug and play} infrastructure primitive for ROS and MCAP users. To
    facilitate this, dp3 accepts MCAP files as its input format, stores data in
    MCAP format in the leaves of the tree, and streams MCAP out during playback.
    The messages recorded by the user are repackaged and compressed, but not
    otherwise transformed from their original encoding, making dp3’s output
    compatible with any internal log tooling the user has developed. Once dp3’s
    supported set of message encodings is expanded, it should be fully usable as a
    storage component by any ROS user.

    The main contributions of dp3 are: (1) A single system for playback,
    search, and analytics of multimodal logs. While these purposes are
    individually satisfiable by other solutions, dp3 consolidates them under
    one platform. (2) A database system leveraging the MCAP file format for
    data storage. To our knowledge dp3 is the only such system. (3) A
    stream-oriented query language for expression of merges and as-of joins.
    (4) A database that directly targets compatibility with existing ROS
    infrastructure.

    The first section of this paper will elaborate on some of the design goals and
    high-level context surrounding the project. The second section will describe
    the implementation of key components in greater detail.

    \section{Background}
    \subsection{Multimodal log management}
    In robotics, it is common to use the word \q{logging} to refer to the process of
    writing multimodal diagnostic messages to network or disk for followup
    analysis. Perhaps by intent, the term serves to emphasize the parallels with
    traditional server application logging, i.e applications log text, robots log
    point clouds.

    Server application log management platforms are plentiful and follow
    well-established patterns. Some examples of these include Sumo Logic
    \cite{sumologic}, Datadog logging \cite{datadoglog}, and the log management
    interfaces of the three major cloud providers: Google Cloud Logging \cite{gcplog}, AWS
    Cloudwatch Logs \cite{cloudwatch}, and Azure Monitor Logs \cite{azlog}.

    From the user’s point of view, these log management solutions are mostly
    interchangeable. The core interface consists of, (1) Search controls:
    dropdowns for error levels, date picking, application names, tags, and a
    text box for \q{advanced} searches in a limited query language; (2) A
    listing of log lines matching the search results in descending timestamp
    order; (3) A \q{view in context} feature, usually accessed by
    right-clicking a particular log line in the listing. This feature will
    show the log line of interest highlighted (or \q{pinned}) in a wider scope,
    perhaps by dropping the search restriction. For instance if the original
    query was restricted to a single application, the \q{context} may now
    include all applications. This feature is used for tracing cascading
    failures and events across services.

    Users of multimodal logs have a similar set of requirements, with additional
    challenges.

    First, there are more kinds of content-based search, such as geographical
    search, image similarity search, or other kinds of search unanticipated by
    the database designer (i.e UDFs), and perhaps a more frequent need to
    correlate multiple signals in a single search. The need for timestamp-based
    search controls and filtering on high-level dimensions remains as great as
    before.

    Second, the meaning of \q{listing log lines} or \q{filling a screen with
    logs} becomes a lot less clear and is an interesting question in user
    interface design. Log lines can still be listed as JSON, with bytes-valued
    fields rendered in base64 or similar – but more creative approaches can
    likely do better by incorporating images, pointcloud visualizations, maps,
    audio, and more.


    Third, text rendering is clearly insufficient for the \q{view in context}
    function, which now represents replaying \q{what the robot saw} at the
    associated time. Instead of a pinned message in a larger resultset, \q{view
    in context} is implemented using a complex visualization application such as
    Foxglove \cite{foxglove}.

    Beyond these parallels, there are other points of divergence:

    \textbf{Different organizational profile.} Server application logs are
    generally browsed only by infrastructure teams and application
    administrators. The read activity on those databases is sparse and high
    latencies are tolerable. In a robotics development organization, a
    multimodal log database is a business-critical, user-facing service and
    may be depended on by most of the development team outside of
    infrastructure. High latencies in playback and search are a direct
    impediment to product development speed.

    \textbf{Different scales of volume.} Depending on what kind of data the user
    is logging, multimodal data volumes can be orders of magnitude larger
    than server application log volumes for the same organization. Large
    producers may produce hundreds of terabytes or petabytes of logs per day.
    This makes it relatively more important for administrators to have a
    clear understanding of data access patterns, and granular control over
    data retention.

    \textbf{Support for distributed compute.} Typical server log databases do
    not expect you to run distributed compute jobs on your log data and are not
    designed with that use-case in mind, whereas in multimodal log management,
    it is a core workload.

    \textbf{Support for statistical aggregates.} In many cases, multimodal logs
    are similar to metrics. It can be attractive to be able to plot statistical
    aggregates such as quantiles or averages of field values over selected
    windows of time. The APM portions of Datadog enable this kind of operation
    at low latency even over trillions of points. To drive a responsive
    interface these aggregations must rely on precomputation.

    While there are undoubtedly large organizations where these concerns apply
    to application log management, in multimodal logging they become relevant
    at much smaller levels of company scale, when infrastructure teams are less
    developed.

    \subsection{ROS and MCAP}
    dp3 integrates closely with two existing software projects: the ROS (Robot
    Operating System) ecosystem, and the MCAP log file format.

    ROS is a software framework frequently used in academia and industry for
    developing robotics applications. A ROS-based system is built from
    \q{nodes}, which are independent processes that communicate via \q{topics}
    in pub/sub fashion over a message bus. Nodes may be attached to sensor
    drivers or process the outputs of other nodes. ROS provides various
    pre-built nodes for common kinds of robotics operations in Python and C++,
    along with extension support for writing your own nodes.

    The ROS project has recently completed a transition from ROS1 to ROS2
    \cite{ROS2}. One important aspect of this transition was a switch from a
    prescribed message serialization format (ros1msg) to a pluggable recording
    architecture where the user is free to choose whatever serialization they want.
    For example CDR \cite{cdr}, protobuf \cite{protobuf}, and flatbuffers
    \cite{flatbuffers} are common choices. Flexible message serialization is a
    benefit for users but a complication for tooling authors, who must support
    whatever format the user chooses instead of focusing on just one.

    MCAP is a container format for multimodal logging developed at Foxglove,
    intended to support the pluggable serialization goals of ROS2 while still
    standardizing format features such as time-based indexing, chunk compression,
    integrity checks, and efficient remote summarization. The goal of MCAP is to
    enable a shared ecosystem of multimodal log tooling in spite of the variation
    among organizations in message serialization choice. MCAP has been the
    default log format of ROS2 since version \q{Iron Irwini}, released in May
    2023 \cite{irwini}.

    dp3 targets integration with MCAP because the ROS ecosystem, and by extension
    the MCAP ecosystem, is significantly populated by educational applications
    and early-stage startups in position to try new approaches to data
    management. Accordingly, dp3 is built around MCAP’s domain model:

    A \textbf{schema} is a named schema definition in a format consistent with the
    encoding format of the log’s messages.

    A \textbf{channel} is a logical stream of messages within a log file. A channel is
    associated with a schema, as well as a topic.

    A \textbf{topic} is a name for a logical data stream, for instance \q{/images}. Each
    channel has exactly one topic, multiple channels may have the same topic, and
    channels with the same topic may have different schemas (unusual in a single
    recording but common over time as an organization’s schemas evolve).

    A \textbf{message} associates a channel with an array of bytes, which encodes a
    message in the format described by the schema of the channel.

    \subsection{As-of joins}
    A central aspect of multimodal log search workloads is a similar operation to
    what some SQL dialects call an \q{as-of join}. The idea of an as-of join is
    to join records in one table with those in another based on a sequential or
    temporal relationship - for instance, join records in one table with the
    immediately preceding or succeeding records in another table, based on a
    timestamp column.

    As a result of the message-oriented architecture described in the previous
    section, these kinds of queries are common in robotics. For an example query,
    consider working on a self-driving car and needing to locate recent instances
    where the car was taking a left hand turn in the rain, with more than 10
    pedestrians present in an intersection, in order to debug some misbehavior
    that has been noticed. The log messages that would indicate the left turn,
    the rain, and the number of tracked pedestrians will likely occur on separate
    topics, at unaligned timestamps and different logging frequencies. This
    makes the desired result the cases where variously-defined \q{adjacent}
    readings on these topics match your search criteria.

    As-of joins are a nonstandard SQL feature, and are not supported in all
    analytics databases. In dialects where they are unsupported they are
    generally emulatable with more complex SQL syntax. None of BigQuery
    \cite{bigquery}, Redshift \cite{redshift}, or Azure Synapse \cite{synapse},
    support them.  DuckDB \cite{duckdb}, Clickhouse \cite{clickhouse}, and
    Snowflake \cite{snowflake} all do.

    Regardless of whether an engine has explicit support for as-of joins, there
    are complexities with putting SQL in front of the user and telling them to
    write complex N-way as-of joins. First, while the robotics developer may have
    some basic SQL experience, they are not a SQL expert, and particularly if the
    dialect does not support explicit as-of joins the queries can become
    extremely complex. Although one workable option is to hire SQL experts to
    interface between the database and the robotics developer, one of dp3's
    experimental goals is to avoid this solution and attempt instead to empower
    the robotics developer directly. Second, analytics databases are not
    general-purpose databases and often have non-obvious performance footguns and
    tricks that you need to understand to really use them effectively. A typical
    user with basic SQL experience has usually gained that experience from a
    general purpose RDBMS, and may be aggravated to find that the SQL
    implementation of an analytics DBMS comes with a host of unwritten guidance
    along the lines of \q{don't use joins} or \q{manually push down where
    clauses}, or that join algorithms must be directly specified. SQL knowledge
    is portable to a degree, but perhaps less than might initially appear.

    In dp3 we target a narrower language feature set with explicit focus on
    ergonomic as-of join support. While we compile these queries to our own
    execution logic, the binding of the language to dp3 is not very tight, and
    there would be no conceptual barrier to instead compiling to e.g Clickhouse
    SQL in a way that transparently handles the quirks of Clickhouse for the
    user and results in a statement that should execute efficiently. This could
    be a useful extension strategy for a platform built on dp3.

    \section{Implementation}
    \subsection{Program architecture}
    \begin{figure}
      \begin{forest}
        for tree={
          draw,
          semithick,
          minimum height = 3ex,
          minimum width = 2em,
          grow = south,
          forked edge,
          s sep = 3mm,
          l sep = 6mm,
          fork sep = 3mm,
        }
        [router
        [treemgr
        [rootmap
        [sql]]
        [versionstore
        [sql]]
        [walmgr
        [disk]]
        [nodestore
        [cache]
        [S3]]]]
      \end{forest}
      \caption{Component structure of dp3}
      \label{modstructure}
    \end{figure}

    The component structure of dp3 is depicted in \autoref{modstructure}. This
    section will briefly outline the main data flows, and the following sections
    will explain components in greater detail.

    \subsubsection{Write path}
    On the write path, the router receives an MCAP data stream and passes it
    into the tree manager, which splits it into per-topic MCAP-format buffers
    in memory, while parsing the messages and computing statistics. Buffers are
    flushed into the WAL as a configurable size threshold or on full
    consumption of the input.

    The WAL manager monitors data outstanding in WAL for both size and time
    since last update. When the outstanding data of a stream exceeds a size
    threshold, or when the stream has not received a new write within a
    configurable amount of time, leaf pages from the WAL are merged into a
    copy-on-write tree overlay, and written into storage as a single object.
    This helps to decouple storage write dimensions from the user’s recording
    pattern: if the user is recording a lot of small files, we will batch data
    from multiple input files into larger storage writes.

    Following a successful write to storage, the location of the root, along with a
    version from the version store, and the server’s timestamp, are written to the
    rootmap database.

    dp3 does not support updates, and writes to storage must be serialized
    per-tree.

    \subsubsection{Read path}
    On the read path, the router receives a request as a string in the dp3
    query language. The query is parsed and transformed into an iterator-style
    execution plan, with a root location retrieved from the tree manager for
    each topic involved in the query. Scan nodes are instantiated with a
    timerange and a filter expression and are executed by walking the relevant
    tree leaves and applying the filter. Leaf nodes are structured as linked
    lists and sometimes have multiple \q{insert} and \q{delete range}
    operations, which the scan operator resolves into a set of file read and
    merge operations.

    Node lookups go through the nodestore, with a read-through byte-capacity
    bounded LRU cache of inner nodes. If the important inner nodes of the tree are
    in the cache, then the only reads from storage are those for the leaf data.

    After streaming through the tree of query operators, records are written out
    to the user in a single topic-multiplexed MCAP data stream.

    \subsection{Public API} \label{api}
    The public API of dp3 consists of the following core operations:

    \textbf{Import(database string, producer string, objectID string)} imports an
    object from storage on behalf of a producer by telling dp3 a location from
    which to read. The term \q{producer} is intentionally vague: it is up to the
    user to decide whether to associate producers with \q{devices} or
    \q{simulation runs} or another internal concept. dp3 supports multiple
    logical databases within a single instance, and operations must be scoped
    with a database.

    \textbf{Query(database string, query string)} evaluates the query and
    returns results in MCAP format.

    \textbf{StatRange(database string, producer string, topic string, start
    uint64, end uint64, granularity int)} returns a range of field-level
    statistical aggregations at a granularity at least as granular as the one
    requested. These statistics are stored in the inner nodes of the tree, so
    under optimal circumstances StatRange queries should mostly be served from
    RAM.

    \textbf{Delete(database string, producer string, topic string, start uint64,
    end uint64)} performs a logical deletion of data between two timestamps.
    Physically this is an insert of a \q{mask} rather than a true delete.
    Physical deletion is handled by storage retention policies only.

    \textbf{Truncate(database string, producer string, topic string, timestamp
    uint64)} perform a logical data truncation at a particular timestamp. This
    makes data inserted prior to that timestamp unreachable by read requests. To
    prevent reads from accessing deleted files, it is necessary to truncate
    tables prior to the time when storage retention policies would delete their
    objects.

    \textbf{Messages(database string, producer string, topics
    map[string]uint64), start uint64, end uint64} returns a merged result of
    messages on the provided set of topics in timestamp order. The values of
    the passed map are minimum tree versions, and the HTTP headers in the
    response of the method indicate the versions of the topics in the response.
    This API is used to emulate "log tailing", by polling for messages since
    the last tree version seen. We will probably remove this once we decide how
    to incorporate the tailing functionality into the query executor.

    \subsection{Multigranular tree}
    The core storage structure of dp3 is a versioned, time-partitioned,
    copy-on-write tree. One tree is stored for each producer/topic combination.
    Internally we refer to this combination as a \q{table}, however the user deals
    only with producers and topics. Trees are constructed based on a start and end
    time, branching factor, and target leaf width in seconds. The constructor then
    picks a height that results in a tree satisfying the dimensional constraints.
    The height obtained by a default tree is 5.

    \begin{figure}
      \begin{tabular}{ |c|c|c| }
        \hline
        bytes & value & description \\
        \hline
        8 & uint64 & version \\
        8 & uint64 & offset \\
        8 & uint64 & length \\
        \hline
      \end{tabular}
      \caption{Structure of a node ID}
      \label{nodeidstructure}
    \end{figure}

    The nodes of the tree are divided between inner and leaf nodes. All data is
    stored in the leaf nodes in MCAP format, with the inner nodes responsible for
    storing time bounds along with an array of children, each associated with
    statistics when present. Within dp3 and in the child arrays of inner nodes,
    nodes are addressed with a 24-byte ID, the structure of which is depicted in
    \autoref{nodeidstructure}.

    The tree module implements three core write-related methods: Insert,
    DeleteRange, and Merge. Insert and DeleteRange are both physical inserts. They
    each construct a path from a root to a leaf and serialize that path to the WAL.
    When the WAL manager chooses to sync the WAL data to storage, the Merge method
    is called on these partial trees, along with the relevant inner node structure
    of the destination tree in storage, to form a copy-on-write overlay.

    \subsubsection{Truncation}
    We support efficient, synchronous, logical deletion from the tail of the tree
    of data before a timestamp. This is implemented by looking up the tree version
    at that timestamp in the rootmap, and setting the table’s minimum version to
    that version plus one. Subsequent scan operations are made aware of the minimum
    version and ignore the data.

    Truncating the table ensures that no reader will access deleted objects after
    storage lifecycle policies physically reap the files. Truncation must be
    scheduled by the administrator to stay ahead of the lifecycle policy-initiated
    deletion.

    \subsubsection{Range-based deletion and overlapping inserts}
    In addition to truncation from the tail of the tree, we support the ability to
    delete messages from an arbitrary time range. Range-based deletion is
    accomplished with a combination of two strategies. Any tree node impacted in a
    deletion scenario is either fully or partly covered. If a node is fully covered
    by the deletion, the partial tree created by the DeleteRange function will omit
    the node and include a tombstone indicator in the corresponding location in its
    parent’s child array. This rule can be applied down the tree until we are left
    with the case of partially-split leaf nodes to resolve.

    We implement leaf nodes using a linked list structure. The node has a header
    prior to the start of the data that includes an ancestor node ID and version,
    as well as an ancestor delete and end time.

    If the bytes of the ancestor node ID are all zero, there is no ancestor.
    Otherwise, the header indicates that readers must merge the ancestor node’s
    data with the data of the node in hand. If the deletion end time is
    nonzero, that range is interpreted by the reader as a mask to apply over
    the merge of ancestor data.

    Overlapping inserts use the same linked list structure and simply leave the
    deletion range zero. If we receive a write for a leaf that already has data on
    it, we handle that as a physical list append rather than a physical merge of
    message data.

    Consequently our tree merge operation never needs to read message data from
    storage, only inner nodes. However, we are subject to a form of \q{bloat} from
    heavy overwrites until we implement some kind of compaction process.

    \subsubsection{Storage object structure}
    Our writes to both the WAL and storage are based around an internal structure
    called a \q{memtree}. The memtree is a linked structure of node pointers amenable
    to in-place manipulation. The memtree’s byte serialization is the structure of
    both our storage objects and the \q{data} portion of the WAL insert records.

    In the memtree, nodes are associated with random, temporary IDs. To serialize
    the memtree to disk, we walk it breadth-first and collect a path of nodes. We
    reverse the path and serialize the nodes to an output buffer in
    reverse-dependency order, so that the offset and length used in construction of
    the real node IDs are known at the time when they need to be recorded into the
    child arrays of their parents.

    The final node serialized to the output is the root node. After the root node
    is serialized, we write its ID to the final 24 bytes of the output, enabling a
    standalone reader to interpret the file by reading the last 24 bytes and
    seeking to the root offset from which to start traversing.


    \begin{figure}
      \begin{tabular}{ |c|c|c| }
        \hline
        bytes & value & description \\
        \hline
        1 & uint8 & physical node version \\
        24 & node ID & ancestor node ID \\
        8 & uint64 & ancestor version \\
        8 & uint64 & ancestor delete start \\
        8 & uint64 & ancestor delete end \\
        n & bytes & mcap leaf data \\
        \hline
      \end{tabular}
      \caption{Byte serialization of a leaf node}
      \label{leafstructure}
    \end{figure}

    The byte serialization of a leaf node is depicted in \autoref{leafstructure}.

    The ancestor-related fields may be zero depending on whether there is an
    ancestor and whether the leaf represents a logical insert or delete.

    \begin{figure}
      \begin{tabular}{ |c|c| }
        \hline
        value & description \\
        \hline
        uint64 & start time \\
        uint64 & end time \\
        uint8 & height \\
        $[]$child & children \\
        \hline
      \end{tabular}
      \caption{Structure of an inner node}
      \label{innerstructure}
    \end{figure}

    Our inner nodes are currently serialized as a uint8 physical node version
    followed by the JSON serialization of the inner node structure, which is
    depicted in \autoref{innerstructure}.

    \begin{figure}
      \begin{tabular}{ |c|c| }
        \hline
        value & description \\
        \hline
        node ID & child node ID \\
        uint64 & child version \\
        map[string]statistics & per-schema statistics \\
        \hline
      \end{tabular}
      \caption{Structure of an inner node child array element}
      \label{childstructure}
    \end{figure}

    The structure of an inner node child is depicted in \autoref{childstructure}.

    We anticipate switching the inner nodes from JSON to a packed format before
    long.

    The statistics stored on inner node children are stored per-schema, with
    schemas identified using a sha1 content hash. Storing statistics for multiple
    schemas is necessary because the schemas of our tables can evolve over time,
    including in ways that typical notions of schema evolution would call
    incompatible, meaning a field could (as far as we are concerned) have
    different meanings or types at two different points in time. In order to
    resolve queries and provide useful statistics in this situation, we store
    data on each schema separately and leave it to the consumer to decide what to
    do.

    The statistics themselves are computed for all fields except variable-length
    arrays, and different statistics are supported depending on whether the field
    is numeric or text. The major constraint on the statistics we store is that
    they are associative: it must be possible to derive a correct updated
    statistic from an existing statistic and the new data being inserted, without
    rescanning old data. This property holds for some common statistics such as
    sum, min, max, count, and average (assuming count is also maintained), but
    does not hold for others such as exact quantiles. There are approximate
    alternatives that may be useful for addressing this gap, such as the
    DDSketch \cite{ddsketch}.

    The text statistics we store today are very limited: only min and max are
    supported. However, some useful-seeming structures obey the associative law
    and could be used for text, such as Bloom filters, unique sets, and trigram
    bitmaps.

    The first byte of both the leaf and inner node serializations is the physical
    node version. This is distinct from the \q{version} referred to elsewhere in
    the system, such as in the \q{ancestor version} field of the leaf node. The
    physical node version serves both to evolution of the physical node format
    and indicate to readers whether they are dealing with a leaf or inner node.
    Leaf and inner nodes split the range of the unsigned byte, enabling 128
    possible versions for each.

    \begin{figure}
      \begin{tabular}{ |c|c|c| }
        \hline
        bytes & value & description \\
        \hline
        1 & uint8 & record type \\
        8 & uint64 & data length (\textbf{n}) \\
        \textbf{n} & bytes & record data \\
        4 & uint32 & crc32 of preceding fields \\
        \hline
      \end{tabular}
      \caption{WAL record serialization}
      \label{walstructure}
    \end{figure}

    Storage objects are named with the decimal representation of the tree version
    they correspond to.  WAL manager The write ahead log is a single, append-only
    stream of binary records. The format of a WAL record is depicted in
    \autoref{walstructure}.

    As with storage objects, WAL filenames are decimal integers, and WAL records
    have 24-byte addresses similar to node IDs, which can be resolved to a WAL
    filename, offset, and length.

    They byte identifier of the WAL record takes one of three types, indicating
    how the data should be parsed: \textbf{Insert} contains a serialized partial
    tree covering one leaf node.  \textbf{MergeRequest} contains a list of insert
    record addresses that should be merged as a batch into storage, as well as a
    UUID-format batch ID.  \textbf{MergeComplete} records the completion of a
    merge operation into storage by batch ID.

    On startup, the WAL manager scans all outstanding WAL files in the WAL
    directory, replaying their records through its accounting until it arrives at a
    final state representing unmerged work in the WAL. Thereafter, writes to the
    WAL are accounted for in memory and this state is used for scheduling of
    flushes to storage.

    Garbage collection of the WAL is handled with a process that polls the
    in-memory state of the WAL manager to determine if any files on disk are no
    longer referenced. Any such files are deleted.

    \subsection{Rootmap}
    The rootmap fronts a database of root locations used by the tree manager.
    Every write that is finalized to storage records a write to the rootmap,
    and every read from dp3 (when operating in server mode) does a read from
    the rootmap.  While the rootmap is technically rebuildable from storage
    operations by listing objects and reading root IDs from the final 24 bytes,
    it is essential for the functioning of the database and effectively a
    central point of failure.

    The rootmap interface is fairly straightforward and should be portable to a
    range of database solutions:

    \textbf{Get(database string, producer string, topic string, version uint64)}
    gets a root for in a database a table at a version.

    \textbf{GetLatestByTopic(database string, producer string, topics
    map[string]uint64)} gets the latest roots in a database for a set of tables.
    The passed map of topics indicates minimum tree versions for each topic.

    \textbf{GetHistorical(database string, producer string, topic string)} gets
    all root versions in the database for a single table.

    \textbf{Put(database string, producer string, topic string, version uint64,
    nodeID nodeID)} inserts a new root into the rootmap.

    One dp3 instance can contain multiple logical \q{databases}, allowing users to
    segregate simulation data from real-world data or maintain other such
    divisions. All operations against the rootmap are therefore scoped with a
    logical database identifier.

    Today, dp3 uses SQLite \cite{sqlite} for the rootmap.

    \subsection{Versionstore}
    For a single tree, writes must be versioned in monotonically increasing order
    to support range deletion as well as querying for new data since a prior read.

    The version store is backed by a persistent counter, from which a fixed
    number of versions are reserved at a time. On startup, a dp3 instance
    acquires a lock on the counter and increments it, thus reserving a set of
    versions before another lock must be taken. When the end of the
    reserved range is reached, a new reservation is made. This batching
    minimizes contention over the shared counter.

    This mechanism exists to ensure that within a tree, a single node will only
    record increasing versions for a table. When write clustering support for dp3
    is developed, we will still maintain an invariant that writes for a single
    table flow through only one node at a time. If responsibility for a table
    transitions from one node to another (i.e during resharding), a new
    reservation operation or service restart will be required.

    The versionstore is backed by SQLite today, and is colocated in the same
    database as the rootmap. The only required feature is the ability to lock and
    set the shared counter, which is supported by any mainstream SQL database,
    and activity on it is very sparse. To reduce component sprawl, the
    versionstore and rootmap will likely remain colocated as long as the rootmap
    is also on SQL.

    \subsection{Query language}
    The kinds of searches executed on dp3 are expected to be heavy on multiway
    as-of joins. In natural language, a typical kind of search might be, \q{show me
    the times in the last month where we were taking an unprotected left hand turn,
    in the rain, with dogs and bicycles in the intersection.}

    Each of these criteria (unprotected left, is it raining, what objects are
    tracked) will be stored on different tables, and the timestamps of each table
    will not be in alignment, as the messages will be logged at different
    frequencies. This means the utility of typical equijoins is diminished, and
    query conditions based on relative time proximity or immediate precedence are
    more useful – in this case, locating instances where we see dogs and bicycles
    and the prior instance of /raining reported \q{true}, and the prior instance of
    /unprotected\_left indicated the beginning of a turn. These examples are of
    course simplified.

    In dp3 we have developed a query language for ergonomic expression of as-of
    joins. The dp3 query language includes the following features: (1) Where
    clauses with support for nested fields and typical SQL binary operators. (2)
    Parenthetical groupings. (3) Time-ordered merging of tables. (4) As-of joins,
    based on either a time window, immediate succession, or both. (5) Reversal of
    scan order. (6) Time-based restriction on log time. (7) Limit and offset.

    Currently, all queries must be scoped with a single device. All queries are
    terminated with a semicolon. Some examples of valid queries are listed below. \\

    Scan all messages on a topic.
    \begin{lstlisting}
    from my-robot /topic;
    \end{lstlisting}

    Scan all messages on a topic with a where clause.
    \begin{lstlisting}
    from my-robot /topic as t
    where t.my_field > 10;
    \end{lstlisting}

    Return a time-ordered merge of two topics.
    \begin{lstlisting}
    from my-robot /topic1, topic2;
    \end{lstlisting}

    Restrict from clause with a time range.
    \begin{lstlisting}
    from my-robot between
    "2020-01-01" and "2021-01-01"
    /topic;
    \end{lstlisting}

    Return merged results with conditions on both tables.
    \begin{lstlisting}
    from my-robot
    /topic1 as t1, /topic2 as t2
    where t1.field = "foo"
    or t2.field = "bar";
    \end{lstlisting}

    As-of join with where clause with condition on either child.
    \begin{lstlisting}
    from my-robot
    /topic1 as t1 precedes /topic2 as t2
    by less than 5 seconds
    where t1.field = "foo"
    or t2.field = "bar";
    \end{lstlisting}

    As-of join with immediate modifier.
    \begin{lstlisting}
    from my-robot
    /topic1 as t1
    precedes immediate /topic2 as t2
    where t1.field = "foo"
    or t2.field = "bar";
    \end{lstlisting}

    Reversal of scan order.
    \begin{lstlisting}
    from my-robot /topic desc;
    \end{lstlisting}

    These queries get compiled into a typical iterator-style query execution plan,
    and executed by repeatedly calling Next on a root executor node.

    The current operators we implement are \q{merge}, as-of join, scan, filter,
    limit, and offset. The merge operation is associated with the comma operator
    in the language grammar, and it executes an N-way streaming merge of
    time-ordered children through a binary heap. This is the same way a
    conventional RDBMS may implement an ordered union, when indexes exist on the
    ordering column.

    The as-of join operations (precedes or succeeds keyword) are executed by first
    inverting \q{succeeds} terms to \q{precedes}, and then feeding the two sides
    into an operator that caches the previously-observed LHS value, and evaluates
    on each RHS observation whether the previously-observed LHS and current RHS
    satisfy the as-of condition. A specification of \q{immediate} causes only the
    first RHS value to be returned for each LHS. Without the immediate modifier,
    all matches will be returned.

    \section{Discussion}

    \subsection{Related Work}
    The copy-on-write tree structure of dp3 borrows heavily from btrdb
    \cite{Andersen}, a timeseries database for storing streams of
    high-frequency univariate floats. Comparing the two, btrdb's use of
    univariate floats makes it effectively columnar, and enables it to use
    compression techniques that are not available to dp3, which stores
    multivariate rows with chunked zstd compression.

    Another system that uses a similar storage scheme is Apache Iceberg
    \cite{iceberg}. Iceberg is a table format specification rather than an
    implementation, but the format specifies a height-three copy-on-write tree
    consisting of metadata files, manifest lists, manifest files, and data
    files. The metadata is a JSON file storing a list of \q{snapshots} that
    reference manifest lists. Manifest lists include summary information about
    manifest files, and manifest files contain a list of data files in Avro
    format. Iceberg has built-in support for Parquet, Avro, and ORC format
    data files. As with dp3 and btrdb, it seems logical in deployment scenarios
    to cache the upper nodes of the tree.

    The primary intent of Iceberg is to enable multiple database engines to
    operate against the same data safely at the same time, but as a useful
    side-effect of having data in Iceberg format you gain the ability to use an
    off-the-shelf database engine for free, and the potential for pluggable
    integrations in the future if more databases add support for external Iceberg
    tables, as Snowflake has recently done. In addition, dp3 may at some point
    want to have multiple engines operating at the same time, for instance if we
    were able to integrate with Spark SQL \cite{spark}. These possibilities make Iceberg a
    worthwhile direction to explore as backing storage for dp3. In comparing the
    two, Iceberg stands out for its far more developed SQL and transactional
    capabilities, while dp3 is distinguished in supporting message storage and
    playback in the user's native choice of recording format, without on-the-fly
    transcoding during read. dp3 also has a more granular time index due to
    having a deeper tree.

    The query language and goals of dp3 bear similarity to query languages and
    systems previously developed in the areas of \q{stream databases} and
    \q{sequence databases}, such as CQL \cite{cql}, the GSQL language for the
    Gigascope database \cite{gigascope}, Aurora \cite{aurora}, and SEQUIN for
    the SEQ system \cite{seq}. These languages are broadly SQL-like with a focus on
    support for merging and as-of join operations, although those go by
    different names.

    CQL has a focus on continuous queries over unbounded data streams. CQL
    supports a \q{sliding window join} with the syntax 

    \begin{lstlisting}
    Select Istream(*)
    From S1 [Range 5 seconds], S2 [Range 10 seconds]
    Where S1.A = S2.A
    \end{lstlisting}

    in this example, a tuple from S1 will be sent to the output if it joins with
    a tuple in S2 from within a 10-second sliding window, and a tuple from S2 will
    be sent if it joins with a 5-second sliding window from S1.

    dp3 does not support this kind of bidirectional windowing today, but this
    query is pretty similar to what dp3 would express as

    \begin{lstlisting}
    from my-device
    S1 precedes S2 by less than 10 seconds
    where S1.field = "foo"
    and S2.field = "bar"
    \end{lstlisting}

    The concept of joining columns between tables in dp3 is not presumed to make
    sense today, because the fields of different tables are typically
    incomparable. However, support for this may become important as we develop
    more focus on subqueries.

    Continuous queries are also not supported in dp3 today, however support
    would not require much work. The tree structure makes it very cheap to
    check if there is any new data since the last time observed. This should
    make it simple to offer a \q{tail} operator as an alternative to scan,
    which will run continuously and watch the tree root for new tuples as they
    come in. The rest of the query plan should remain the same.

    In the SEQUIN language for the SEQ database, an as-of join is expressed in
    this way:

    \begin{lstlisting}
    // first define the moving average as a view
    CREATE VIEW MovAvgStockl AS (
    PROJECT avg(C.high) as avghigh
    FROM stock1 c
    OVER \$P-23 TO \$p.);

    // then use the view in the query
    PROJECT A.avghigh - B.high
    FROM MovAvgStockl A,
    Previous(PROJECT D.high
    FROM Stock2 D
    WHERE D.volume > 25,000) B
    WHERE.\$P > 2000;
    \end{lstlisting}

    The \textit{Previous} function returns the most recent hour at a point during
    evaluation when the volume on Stock2 exceeded 25,000. In dp3 we have no support
    for aggregations at this time, so we are unable to compute the moving average
    represented in the view. In dp3, \q{Previous} is implemented with \q{precedes
    immediate}.

    \subsection{Work remaining}
    dp3 is on github \cite{dp3} in partially implemented form. A license is not
    yet chosen. The database is implemented in Go. dp3 compiles to a single
    binary used for both the server and the command-line interface.

    That interface includes various functions for introspecting the storage,
    such as the \q{treeinspect} and \q{walinspect} commands, as well as an
    interactive terminal similar to psql (from Postgres). From the client
    interface it is possible to list tables, view statistics, and run
    interactive queries.

    Communication between client and server is over an HTTP API mapping closely
    to the functions described in \ref{api}. We view this API as fairly ad-hoc
    and temporary, and intend to either firm it up or switch to grpc once we
    are in a better position to evaluate.

    Significant work remains, both in design validation and feature
    development, to get dp3 to a useful state.


    \subsubsection{Design validation}
    The largest outstanding design question relates to dp3’s choice to use MCAP
    as the data format in leaf storage. There are a few benefits to this
    choice: (1) No transcoding required on ingestion or query. (2)
    Automatically compatible with user’s own internal tooling. (3) Streaming
    playback is a row-oriented workload. (4) Automatically compatible with
    common robotics recording workflows

    There are also drawbacks, primarily in the performance department. (1)
    Columnar storage would realize better compression in general. (2) A
    standard columnar format may allow us to leverage an off-the-shelf query
    executor, and could probably support more performant queries for our search
    workload. (3) Building generic systems on MCAP requires supporting query
    operations on multiple different binary encoding formats in place, which is
    tedious to implement and support, and may carry some performance risks.

    The question is also complicated by workload-specific factors. If a
    columnar format is used, it is likely that small \q{row groups} (or
    whatever equivalent concept exists) will be required, to enable queries to
    merge large numbers of tables at once without exceeding available memory.
    Using small row groups will tend to work against the interests of good OLAP
    performance on the data files, obviating some of the benefits we would be
    hoping to get with the columnar format.

    Furthermore, a significant portion of the heavy read volume the database
    will serve for ML purposes, will be image and point-cloud data. In these
    kinds of messages, there is effectively one column: a byte array, possibly
    up to megabytes in size. This means that row-oriented and columnar files
    for these kinds of data are physically pretty similar, so the impact of the
    change to columnar for these cases may not be that great.

    The best way to resolve all these questions seems like an implementation
    spike. Parquet is an obvious candidate for a columnar format to test as it
    would also be compatible with Iceberg. The LanceDB \cite{lance} project is
    another one that is focused on some of the more exotic kinds of searches
    users might be interested in, such as image similarity search. Once we have
    a good sense of how all the options perform, we can choose the best option
    and have a strong rationale.

    A second area where outstanding questions exist is whether the structure we
    have, with the statistics we can attach, is really sufficient to accelerate the
    kinds of queries users want to run to be fast enough for a good user
    experience. To get a better feel for this we will need to incorporate more
    statistics and then get some users to evaluate.

    \subsubsection{Feature development}
    Major feature development priorities are listed below.
    \begin{itemize}
      \item Message parsing support for all the message encoding formats we
        intend to support, not just ros1msg. This includes protobuf and CDR at
        a minimum, and likely flatbuffers as well.
      \item Support in the query language for descending into variable-length
        complex arrays. Today we only parse fixed-length arrays. This is
        important because it is relatively common for data producers to batch
        multiple logical messages into a single message using an array, and
        they will still require the ability to execute queries on that data.
        Likewise, we should (under some circumstances) gather statistics on
        these arrays as well and store those on the inner nodes.
      \item The query language must be extended with a \q{neighbors} keyword or
        similar, for a bidirectional as-of join.
      \item The set of collected statistics needs to be expanded, and also used to
        accelerate search. Today we have a small handful of associative statistics
        (min, max, sum, count) for numeric and text fields. We will want to add
        bloom filters, trigrams, possibly distinct sets. Ways of accelerating
        spatial search would also be useful, and in many cases may correlate well
        with time. However, this will require a way for the user to tell us what
        data is spatial data, which we currently lack.
      \item dp3 is currently single-node only. While supporting clustered reads is
        simple, supporting multiple writers will require some design to ensure that
        only one node writes to a table at a time.
    \end{itemize}

    \section*{Acknowledgements}
    Thanks to James Smith for review.

    \printbibliography

  \end{document}
